\documentclass[twocolumn,A4]{article}

% Fontes (XeLaTeX/LuaLaTeX/Tectonic)
\usepackage{fontspec}
% Se paper.tex está em paper/ e fonts/ na raiz do repo:
\setmainfont[
  Path = ../fonts/,
  Extension = .ttf,
  UprightFont = Gilroy-Light,
  BoldFont = Gilroy-Bold,
  ItalicFont = Gilroy-MediumItalic,
  BoldItalicFont = Gilroy-BoldItalic
]{Gilroy}

\usepackage[brazil]{babel}
\usepackage{amsmath}
\usepackage{amssymb}
\usepackage{graphicx}
\usepackage{microtype}
\usepackage{placeins}


\usepackage{fancyhdr}
\usepackage{lastpage}

\pagestyle{fancy}
\fancyhf{}

% Aumenta o espaço reservado para rodapé
\setlength{\footskip}{30pt}

% Rodapé pequeno e multilinha à esquerda
\fancyfoot[L]{%
\scriptsize
Fonte dos dados: Berkeley Earth (CC-BY-NC).
Processamento: Python + NumPy + Pandas.
Workflow: GNU Make + Tectonic.
IAG-USP (2026).
}

% Numeração pequena à direita
\fancyfoot[R]{\scriptsize \thepage/\pageref{LastPage}}

\renewcommand{\headrulewidth}{0pt}
\renewcommand{\footrulewidth}{0.3pt}



% Citações
\usepackage[round,authoryear,sort]{natbib}

% Hyperlinks (carregar por último para evitar conflitos e garantir links em citações)
\usepackage{hyperref}
\hypersetup{
  colorlinks=true,
  allcolors=blue,
  breaklinks=true,
  pdftitle={Mudança na temperatura média de países nos últimos cinco anos},
  pdfauthor={Iago Guilherme}
}

% Variáveis geradas pelo código
\input{variaveis/n_paises.tex}
\input{variaveis/paises_extremos.tex}




\begin{document}

\title{Mudança na temperatura média de países nos últimos cinco anos}

\author{
  Iago Guilherme\textsuperscript{1}\thanks{\href{https://github.com/IagoGuilherme}{github.com/IagoGuilherme}}
  \\[0.15cm]
  {\small \textsuperscript{1}Instituto de Astronomia, Geofísica e Ciências Atmosféricas (IAG-USP), Brasil}
  \\[0.25cm]
  {\small \textbf{Instrutor:} \href{https://github.com/leouieda}{Leonardo Uieda}}
  \\[-0.05cm]
  {\small \textbf{Monitores:} \href{https://github.com/arthursmacedo}{Arthur Siqueira de Macêdo}, \href{https://github.com/YagoMCastro}{Yago Moreira Castro}}
}

\date{\today}
\maketitle

\begin{abstract}
\textbf{Mini Abstract (English).}
This study analyzes recent variations in mean monthly surface temperature for \NPaises{} countries.
A linear model is fitted to the last five years of observations to estimate the annual temperature change rate for each country.
The results highlight regional differences in recent trends and enable quantitative comparisons among countries.
\end{abstract}

\section{Introdução}

A variabilidade da temperatura média global constitui um dos principais indicadores das mudanças climáticas em curso. Registros paleoclimáticos e instrumentais indicam que a temperatura média da Terra tem apresentado tendência de aumento ao longo do Holoceno, com intensificação nas últimas décadas \citep{Osman2021}. Entretanto, apesar da tendência global de aquecimento, a distribuição espacial dessas variações não é uniforme, refletindo a influência combinada de forçantes radiativas, variabilidade atmosférica regional e processos oceânicos.

Análises em escala nacional permitem identificar padrões regionais e contrastes espaciais relevantes. Embora estudos de longo prazo sejam fundamentais para compreender tendências climáticas estruturais, investigações em janelas temporais mais curtas — como os últimos cinco anos — podem revelar variações recentes associadas a eventos extremos, oscilações climáticas e padrões atmosféricos transientes.

Neste trabalho, investigamos a variação recente da temperatura média mensal em diferentes países, ajustando um modelo linear às séries temporais correspondentes ao período mais recente disponível. O objetivo é quantificar e comparar as taxas anuais de variação entre países, destacando contrastes regionais e padrões espaciais emergentes.

\section{Metodologia}

Seja a taxa de variação da temperatura $\alpha$, medida em graus Celsius por ano. Assumimos que, em escala temporal recente, a evolução da temperatura média $T$ pode ser aproximada por um modelo linear em função do tempo $t$:

\begin{equation}
  T(t) = t\alpha + \beta
  \label{eq:linear}
\end{equation}

\noindent
em que $\beta$ representa o coeficiente linear do modelo. A equação \ref{eq:linear} foi ajustada às séries temporais correspondentes aos últimos cinco anos para cada país, permitindo a estimativa da taxa de variação $\alpha$.

\section{Resultados}

Os maiores valores estimados ocorreram em \PaisesMaiores{}, enquanto os menores ocorreram em \PaisesMenores{}.
Foram analisadas séries temporais de temperatura média mensal para \NPaises{} países, estimando-se a taxa anual de variação ($\alpha$) a partir dos últimos cinco anos de dados.

Os maiores valores estimados ocorreram em \PaisesMaiores{}, enquanto os menores ocorreram em \PaisesMenores{} (Figura~\ref{fig:variacao}). A distribuição espacial das taxas é mostrada na Figura~\ref{fig:mapa_variacao}.


% Figura 1: barras (mantém no fluxo)
\begin{figure}[htb]
  \centering
  \includegraphics[width=\linewidth]{../figuras/taxas_variacao.png}
  \caption{Cinco maiores e menores taxas de variação no conjunto de dados analisado.}
  \label{fig:variacao}
\end{figure}

A Figura~\ref{fig:variacao} apresenta os cinco países com maiores taxas positivas e os cinco com menores taxas no período considerado.

% Figura 2: mapa ocupando ~metade da página na vertical (altura controlada)
% "height=0.5\textheight" força a imagem a ocupar metade da altura do texto.
% "keepaspectratio" evita distorção.
\begin{figure*}[t]
  \centering
  \includegraphics[width=\textwidth,height=0.5\textheight,keepaspectratio]{../figuras/mapa_variacao.png}
  \caption{Mapa mundial da taxa de variação de temperatura (°C/ano) estimada para os últimos cinco anos. Países sem dados ou sem correspondência no mapa estão em cinza.}
  \label{fig:mapa_variacao}
\end{figure*}

\FloatBarrier



\section{Conclusões}

A análise realizada para \NPaises{} países indica predominância de tendências positivas na variação recente da temperatura média. A maior taxa estimada foi de \MaiorTaxa~°C/ano, enquanto a menor atingiu \MenorTaxa~°C/ano. A média global das taxas estimadas foi de aproximadamente \MediaTaxa~°C/ano.

A distribuição espacial das estimativas revela heterogeneidade regional significativa. Embora o sinal médio global seja predominantemente positivo, indicando aquecimento recente em grande parte dos países analisados, observam-se também tendências negativas em determinadas regiões, possivelmente associadas à variabilidade climática interanual.

Cabe ressaltar que o período analisado — correspondente aos últimos cinco anos — representa uma janela temporal relativamente curta para inferências climáticas estruturais. Ainda assim, as estimativas obtidas evidenciam que variações recentes podem apresentar magnitudes relevantes em escala anual.

O workflow adotado, que integra aquisição automática de dados, processamento, geração de figuras e compilação do artigo, assegura reprodutibilidade computacional e permite a atualização imediata dos resultados com a incorporação de novos dados.


\newpage
\section*{Agradecimentos}

Este trabalho foi desenvolvido no contexto do Curso de Verão do IAG-USP (2026), sob orientação do Prof. \href{https://www.leouieda.com/}{Leonardo Uieda} (\href{https://github.com/leouieda}{GitHub}), com apoio dos monitores \href{https://github.com/arthursmacedo}{Arthur Siqueira de Macêdo} e \href{https://github.com/YagoMCastro}{Yago Moreira Castro}.

\bibliographystyle{plainnat}
\bibliography{referencias}

\end{document}
